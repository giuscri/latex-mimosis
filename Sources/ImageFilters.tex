%%%%%%%%%%%%%%%%%%%%%%%%%%%%%%%%%%%%%%%%%%%%%%%%%%%%%%%%%%%%%%%%%%%%%%%%
\chapter{Image filters as a defense against adversarial input}
%%%%%%%%%%%%%%%%%%%%%%%%%%%%%%%%%%%%%%%%%%%%%%%%%%%%%%%%%%%%%%%%%%%%%%%%

In this chapter we'll test a couple of image filters against
adversarial input. The intuition behind the idea of filtering the image
is that to forge an adversarial input the attacker will try to put
little noise distributed in the image. As these filters highlight the
\emph{important} features of an image we hope they cut out the noise
introduced by the attacker.

For each of the filters tested in this chapter we can set its
parameters such that it can be more or less destructive in regards of
the original information. Intuitively the more invasive the filter the
better will be the defense. Unfortunately the better the defense the
more the lost accuracy too: two different images can be confused as the
same image for the model as the information that distiguished the two
images is now potentially lost. That means that the \emph{best} filter
will be the one that provides the best defense given the accuracy lost
by the model is not \emph{too much}.

<<SHOULDNT WE WRITE AN ATTACK CHAPTER?>> As a baseline we performed an
attack of Fast Gradient Sign with its parameter $\eta$ set to 0.1. We
chose that value for $\eta$ as it's a median value between a
non-effective attack -- when $\eta$ equals 0 -- and an attack that's
mostly unstoppable -- when $\eta$ is greater than 0.25.

<<INSERT 3 IMAGES OF THE SAME INPUT AS ETA INCREASES>>

In our \emph{gray-box} setting we measured an accuracy on the MNIST
test set of 97.39\% and a probability of success for the attacker of
81\%. Again, we expected these quantities to go down as we increase the
\emph{power} of the filter technique: while we try to stop the attacker
we have to reduce the accuracy of the model too.

\section{Principal Component Analysis}

<<WRITE GENERAL STUFF ON PCA>>

We tried the following setting for PCA: retain all the components of an
image (784 pixels), 331 of them, 100 of them, 20 and 10 components.

\begin{itemize}
  \item Retaining all the 784 components: accuracy 97.38\%, adversarial
    success score 82\%.
  \item Retaining 331 principal components: accuracy 97.52\%,
    adversarial success score 71\%.
  \item Retaining 100 principal components: accuracy 97.48\%,
    adversarial success score 44\%.
  \item Retaining 20 principal components: accuracy 94.18\%,
    adversarial success score 23\%.
  \item Retaining 10 principal components: accuracy 83\%, adversarial
    success score 22\%.
\end{itemize}

Of all these setting the best result is achieved when we retained 100
components. The accuracy on the MNIST test set is pretty the same (in
fact, it's even increased with the help of the filter) and the
adversarial success score is halved: from the original probability of
81\% to 44\%. When we're comparing PCA with other image filters we'll
consider this setting of 100 components.
