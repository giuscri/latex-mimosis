\chapter*{Introduction}

This work is about how neural networks can be improved from a
resiliency standpoint, against malicious inputs. Many researchers are
doing serious work (cite papers here) both by finding novel forms of
defenses but also by crafting new kind of inputs that defeat the
state-of-the-art defenses. This thesis tries to do a small step in the
former direction, by trying to measure the performance of a defense
technique.

The work that we're using as a starting point is based on is
\cite{bhagoji2018enhancing}. In that paper, researchers use a technique
called Principal Component Analysis to derive a filter for images. It
turns out that applying that filter on each image before the
classification reduces the rate of success for an attacker that
provides input specifically forged to make the model misclassify it. We
reproduced those results and swapped Principal Components Analysis with
other similar dimensionality reductions techniques. The idea is that as
there's nothing intrinsically special about Principal Components
Analysis hence maybe other filters are better suited for the task.

The work is organized as follows: in Chapter \ref{ch:background} we
provide some background material about Machine Learning in general,
Neural Networks and Adversarial Examples. Chapter
\ref{ch:tools-and-libraries} is about the technological part of this
work --- the tools that we used to perform our experiments. Finally,
Chapter \ref{ch:implementation-of-robust-networks} is about the actual
experiments and measurements, comparing the performance of a Principal
Components Analysis filter to the other filters.
