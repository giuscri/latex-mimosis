\documentclass{mimosis}

\usepackage{metalogo}
\setlength{\parindent}{0pt}

%%%%%%%%%%%%%%%%%%%%%%%%%%%%%%%%%%%%%%%%%%%%%%%%%%%%%%%%%%%%%%%%%%%%%%%%
% Some of my favourite personal adjustments
%%%%%%%%%%%%%%%%%%%%%%%%%%%%%%%%%%%%%%%%%%%%%%%%%%%%%%%%%%%%%%%%%%%%%%%%
%
% These are the adjustments that I consider necessary for typesetting
% a nice thesis. However, they are *not* included in the template, as
% I do not want to force you to use them.

% This ensures that I am able to typeset bold font in table while still aligning the numbers
% correctly.
\usepackage{etoolbox}

\usepackage[binary-units=true]{siunitx}
\DeclareSIUnit\px{px}

\sisetup{%
  detect-all           = true,
  detect-family        = true,
  detect-mode          = true,
  detect-shape         = true,
  detect-weight        = true,
  detect-inline-weight = math,
}

%%%%%%%%%%%%%%%%%%%%%%%%%%%%%%%%%%%%%%%%%%%%%%%%%%%%%%%%%%%%%%%%%%%%%%%%
% Hyperlinks & bookmarks
%%%%%%%%%%%%%%%%%%%%%%%%%%%%%%%%%%%%%%%%%%%%%%%%%%%%%%%%%%%%%%%%%%%%%%%%

\usepackage[%
  colorlinks = true,
  citecolor  = RoyalBlue,
  linkcolor  = RoyalBlue,
  urlcolor   = RoyalBlue,
  ]{hyperref}

\usepackage{bookmark}

%%%%%%%%%%%%%%%%%%%%%%%%%%%%%%%%%%%%%%%%%%%%%%%%%%%%%%%%%%%%%%%%%%%%%%%%
% Bibliography
%%%%%%%%%%%%%%%%%%%%%%%%%%%%%%%%%%%%%%%%%%%%%%%%%%%%%%%%%%%%%%%%%%%%%%%%
%
% I like the bibliography to be extremely plain, showing only a numeric
% identifier and citing everything in simple brackets. The first names,
% if present, will be initialized. DOIs and URLs will be preserved.

\usepackage[%
  autocite     = plain,
  backend      = bibtex,
  doi          = true,
  url          = true,
  giveninits   = true,
  hyperref     = true,
  maxbibnames  = 99,
  maxcitenames = 99,
  sortcites    = true,
  style        = numeric,
  ]{biblatex}

\input{bibliography-mimosis}
\bibliography{Thesis}

%%%%%%%%%%%%%%%%%%%%%%%%%%%%%%%%%%%%%%%%%%%%%%%%%%%%%%%%%%%%%%%%%%%%%%%%
% Fonts
%%%%%%%%%%%%%%%%%%%%%%%%%%%%%%%%%%%%%%%%%%%%%%%%%%%%%%%%%%%%%%%%%%%%%%%%

\ifxetexorluatex
  \setmainfont{Minion Pro}
\else
  \usepackage[lf]{ebgaramond}
  \usepackage[oldstyle,scale=0.7]{sourcecodepro}
  \singlespacing
\fi

\renewcommand{\th}{\textsuperscript{\textup{th}}\xspace}

\newacronym[description={Principal component analysis}]{PCA}{PCA}{principal component analysis}
\newacronym                                            {SNF}{SNF}{Smith normal form}
\newacronym[description={Topological data analysis}]   {TDA}{TDA}{topological data analysis}

\newglossaryentry{LaTeX}{%
  name        = {\LaTeX},
  description = {A document preparation system},
  sort        = {LaTeX},
}

\newglossaryentry{Real numbers}{%
  name        = {$\real$},
  description = {The set of real numbers},
  sort        = {Real numbers},
}

\makeindex
\makeglossaries

%%%%%%%%%%%%%%%%%%%%%%%%%%%%%%%%%%%%%%%%%%%%%%%%%%%%%%%%%%%%%%%%%%%%%%%%
% Incipit
%%%%%%%%%%%%%%%%%%%%%%%%%%%%%%%%%%%%%%%%%%%%%%%%%%%%%%%%%%%%%%%%%%%%%%%%

\title{\texttt{latex-mimosis}}
\subtitle{A minimal, modern \LaTeX{} package for typesetting your thesis}
\author{Bastian Rieck}

\begin{document}

\frontmatter
  \begin{titlepage}
  \vspace*{5cm}
  \makeatletter
  \begin{center}
    \begin{Huge}
      \@title
    \end{Huge}\\[0.1cm]
    %
    \emph{by}\\
    \@author
    %
    \vfill
    Advisor: Malchiodi Dario \\
    Co-advisor: Cesa-bianchi Nicolo' \\
    Fall 2018
  \end{center}
  \makeatother
\end{titlepage}

\newpage
\null
\thispagestyle{empty}
\newpage

  \include{Sources/Abstract}

  \tableofcontents

\mainmatter

  \chapter*{Introduction}
\addcontentsline{toc}{chapter}{Introduction}

This work is about how neural networks can be improved from a
resiliency standpoint, against malicious inputs. Many researchers are
doing serious work \cite{papernot2016cleverhans}
\cite{DBLP:journals/corr/KurakinGB16} \cite{carlini2017adversarial}
\cite{meng2017magnet} \cite{yuan2017adversarial} \cite{xu2017feature}
\cite{liao2018defense} both by finding novel forms of defenses but also
by crafting new kind of inputs that defeat the state-of-the-art
defenses. This thesis tries to do a small step in the former direction,
by trying to measure the performance of a defense technique.

The work that we are using as a starting point is based on is
\cite{bhagoji2018enhancing}. In that paper, researchers use a technique
called Principal Component Analysis to derive a filter for images. It
turns out that applying that filter on each image before the
classification reduces the rate of success for an attacker that
provides input specifically forged to make the model misclassify it. We
reproduced those results and swapped Principal Components Analysis with
other similar dimensionality reductions techniques. The idea is that
there is nothing intrinsically special about Principal Components
Analysis hence maybe other filters are better suited for the task.

The work is organized as follows: in Chapter \ref{ch:background} we
provide some background material about Machine Learning in general,
Neural Networks and Adversarial Examples. Chapter
\ref{ch:tools-and-libraries} is about the technological part of this
work --- the tools that we used to perform our experiments. Finally,
Chapter \ref{ch:implementation-of-robust-networks} is about the actual
experiments and measurements, comparing the performance of a Principal
Components Analysis filter to the other filters. The thesis finishes
with a short conclusion that wraps up the work and points out new
directions in which this thesis could be expanded.


% This ensures that the subsequent sections are being included as root
% items in the bookmark structure of your PDF reader.
\bookmarksetup{startatroot}
\backmatter

  \begingroup
    \let\clearpage\relax
    \glsaddall
    \printglossary[type=\acronymtype]
    \newpage
    \printglossary
  \endgroup

  \printindex
  \printbibliography

\end{document}
